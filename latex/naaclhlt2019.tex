%
% File naacl2019.tex
%
%% Based on the style files for ACL 2018 and NAACL 2018, which were
%% Based on the style files for ACL-2015, with some improvements
%%  taken from the NAACL-2016 style
%% Based on the style files for ACL-2014, which were, in turn,
%% based on ACL-2013, ACL-2012, ACL-2011, ACL-2010, ACL-IJCNLP-2009,
%% EACL-2009, IJCNLP-2008...
%% Based on the style files for EACL 2006 by 
%%e.agirre@ehu.es or Sergi.Balari@uab.es
%% and that of ACL 08 by Joakim Nivre and Noah Smith

\documentclass[11pt,a4paper, french]{article}
\usepackage[hyperref]{naaclhlt2019}
\usepackage{times}
\usepackage[utf8]{inputenc}
\usepackage[T1]{fontenc}
\usepackage{latexsym}

\usepackage{url}

\aclfinalcopy % Uncomment this line for the final submission
\def\aclpaperid{***} %  Enter the acl Paper ID here

\setlength\titlebox{5cm}
% You can expand the titlebox if you need extra space
% to show all the authors. Please do not make the titlebox
% smaller than 5cm (the original size); we will check this
% in the camera-ready version and ask you to change it back.

\newcommand\BibTeX{B{\sc ib}\TeX}

\title{Projet INF8460 Automne 2019 }

\author{Samuel Ferron \\
  1843659 \\
  {\tt } \\\And
  Jean-Frédéric Fontaine \\
  1856632 \\
  {\tt} \\\And
  Mathieu B\'eligon \\
  2032839\\
  {\tt } \\}


\date{}

\begin{document}
\maketitle
%\begin{Intro}
 %This document contains the instructions for preparing a camera-ready
 % manuscript for the proceedings of NAACL-HLT 2019. The document itself
 % conforms to its own specifications, and is therefore an example of
 % what your manuscript should look like. These instructions should be
 % used for both papers submitted for review and for final versions of
 %accepted papers.  Authors are asked to conform to all the directions
 % reported in this document.
%\end{Intro}

\section{Introduction}

Salut
pomme

%\section{Introduction}
%
%The following instructions are directed to authors of papers submitted
%to NAACL-HLT 2019 or accepted for publication in its proceedings. All
%authors are required to adhere to these specifications. Authors are
%required to provide a Portable Document Format (PDF) version of their
%papers. \textbf{The proceedings are designed for printing on A4
%paper.}

\section{Méthodologie}

À faire. 

\subsection{Définition de la tâche/objectif}
Le but de l’exercice est de définir un modèle afin de prédire la similarité sémantique entre deux phrases, comme décrit la tâche Semantic Textual Similarity (STS). La motivation de STS est d’être en moyen de construire une représentation du sens d’une phrase. L’hypothèse est que cette représentation peut être utilisé pour diverses applications comme la traduction machine (MT), la question-réponse (QA), la recherche sémantique et bien d’autres. En effet, la comparaison sémantique entre deux phrases en elle même peut sembler une tâche futile, mais la méthodologie que nous avons implémentée pourra être utiliser dans diverses applications. L’ensemble de données avec lequel nous avons travaillés est STSbenchmark, l’ensemble de référence pour STS. L’ensemble contient 5749 d’enregistrements d'entraînement et 1379 de test. Chaque données contient deux phrases, un identifiant et un score de similarité situé entre 0 et 5 seulement pour les données d'entraînement. Un score de 5 indique que les deux phrases sont sémantiquement identique, 0 indique qu’il n’y a pas de relation. 


\subsection{Définition de l'algorithme/méthode/technique}
À faire.

\subsection{Évaluation expérimentale }

À faire. 

\subsection{Ensemble de données, métriques et baselines }

À faire. 


%\begin{table}[t!]
%\begin{center}
%\begin{tabular}{|l|rl|}
%\hline \bf Type of Text & \bf Font Size & \bf Style \\ \hline
%paper title & 15 pt & bold \\
%author names & 12 pt & bold \\
%author affiliation & 12 pt & \\
%the word ``Abstract'' & 12 pt & bold \\
%section titles & 12 pt & bold \\
%document text & 11 pt  &\\
%captions & 10 pt & \\
%abstract text & 10 pt & \\
%bibliography & 10 pt & \\
%footnotes & 9 pt & \\
%\hline
%\end{tabular}
%\end{center}
%\caption{\label{font-table} Font guide. }
%\end{table}

\subsection{Résultats}


\subsection{Discussion}


\section{Travaux connexes}



\section{Travaux futures et conclusion }

\section*{Acknowledgments}
 

%where \verb|naaclhlt2019| corresponds to a naaclhlt2019.bib file.
%\bibliography{naaclhlt2019}
%\bibliographystyle{acl_natbib}

\appendix





\end{document}
